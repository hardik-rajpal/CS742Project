\documentclass{article}
\usepackage[utf8]{inputenc}
\usepackage{amsmath,amsthm,amsfonts,amssymb,amscd}
\usepackage[a4paper,hmargin=0.8in,bottom=1.3in]{geometry}
\usepackage{lastpage,enumerate,fancyhdr,mathrsfs,xcolor,graphicx,listings,hyperref,enumitem}
\author{Hardik Rajpal}
\title{Syntactic Steganography}
\begin{document}
\maketitle
\section{Introduction}
Steganography refers to the concealment of information within a non-secret message or object. The technique has been applied in images and texts with goals of copyright protection and confidentiality. [Put reference papers here]. This project explores a form text steganography that exploits the syntactic redundancies of the English language. Given a sentence, it can be
paraphrased to a different sentence, while retaining the (almost entire) meaning. Given a \textbf{plain document}, one can use the above facts to encode a \textbf{message} into it to produce an \textbf{output document}. To extract the message from the output document, the implementation requires the \textbf{plain document}.
\section{Approaches}
This section highlights the first few approaches that led up to the final implementation.
\subsection{Syntax-Tree Manipulations}
\subsection{Hash-store Encoding}
\subsection{Predictable LLM Paraphrasing}
\section{Review}
\subsection{Steganography Dimensions}
As highlighted in
\href{https://www.researchgate.net/publication/267767675_Syntactic_Bank-based_Linguistic_Steganography_Approach}{this paper},
steganography techniques can be studied with three dimensions. Below I review the project with each dimension:
\subsubsection{Payload Capacity}
It refers to the ratio of hidden information to cover information. The payload capacity is almost as good as methods of [Lexical steganography?], with the recurring example providing
at most 2 bits per sentence. Additionally, the payload capacity is flexible and can be tuned
based on the text provided.
\subsubsection{Robustness}
It refers to the ability of the system to resist against changes in the cover object. While the method has near zero robustness if the cover object is shared in docx form, there exist ways to circumvent this by converting the document... do there though? check ghostscript de-obscuration.
\subsubsection{Imperceptibility}
the potential of the generated stego object to remain indistinguishable from other objects in the same category.
\subsection{Limitations}
\subsection{Takeaways}
\section{References}
\begin{enumerate}
\item \href{https://www.researchgate.net/publication/267767675_Syntactic_Bank-based_Linguistic_Steganography_Approach}{Linguistic Steganography Researchgate}
\end{enumerate}
\end{document}