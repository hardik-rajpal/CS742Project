\documentclass{article}
\usepackage[utf8]{inputenc}
\usepackage{amsmath,amsthm,amsfonts,amssymb,amscd}
\usepackage[a4paper,hmargin=0.8in,bottom=1.3in]{geometry}
\usepackage{lastpage,enumerate,fancyhdr,mathrsfs,xcolor,graphicx,listings,hyperref,enumitem}
\author{Hardik Rajpal}
\title{Syntactic Steganography}
\begin{document}
\maketitle
\section{Introduction}
Steganography refers to the concealment of information within a non-secret message or object. The technique has been applied in images and texts with goals of copyright protection and confidentiality. [Put reference papers here]. This project explores a form text steganography that exploits the syntactic redundancies of the English language. Given a sentence, it can be
paraphrased to a different sentence, while retaining the (almost entire) meaning. Given a \textbf{plain document}, one can use the above facts to encode a \textbf{message} into it to produce an \textbf{output document}. To extract the message from the output document, the implementation requires the \textbf{plain document}.
\section{Approaches}
This section highlights the first few approaches that led up to the final implementation.
\subsection{Syntax-Tree Manipulations}
\subsection{Hash-store Encoding}
\subsection{Predictable LLM Paraphrasing}
\section{Review}
\subsection{Steganography Metrics}
TODO: from that paper.
\subsection{Limitations}
\subsection{Applications}
\section{References}
\begin{enumerate}
\item 
\end{enumerate}
\end{document}