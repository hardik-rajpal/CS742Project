\documentclass{article}
\usepackage[utf8]{inputenc}
\usepackage{amsmath,amsthm,amsfonts,amssymb,amscd}
\usepackage[a4paper,hmargin=0.8in,bottom=1.3in]{geometry}
\usepackage{lastpage,enumerate,fancyhdr,mathrsfs,xcolor,graphicx,listings,hyperref,enumitem}
\author{Hardik Rajpal}
\title{Syntactic Steganography for Document Leak Identification}
\begin{document}
\maketitle
\section{Introduction}
This project seeks to implement the principles of syntactic steganography for identification of the sources of leaks of confidential documents from a protected server. The security model is based on the following assumptions:
\begin{enumerate}
\item Each protected document is stored on the server in .DOCX format.
\item Each user has an identification number, known to the server.
\item The only way to access the documents is to request them from the server after authentication.
\end{enumerate}
In addition to steganography, additional techniques are employed to futher complicate the task of leaking the documents, such as:
\begin{enumerate}
\item Ghostscript 
\end{enumerate}
\section{Reading Notes from \href{https://www.researchgate.net/publication/267767675_Syntactic_Bank-based_Linguistic_Steganography_Approach}{This Paper}}
\begin{enumerate}
\item Abstract
\begin{itemize}
    \item Syntactic bank-based linguistic steganography approach for information security.
    \item Uses the Stanford parser to obtain a syntax tree from sentences.
    \item Use Shannon-Fano coding to compress input messages (user identification in the project's case) in as few bits as possible.
    \item Syntax transformation task: searches the syntax set of the given sentence within the syntax bank and transforms it into a syntax that can represent the secret in the sentence.
    \item The resulting text is still ``innocent-looking,'' and the transformation doesn't affect the semantics.
\item Additionally use a HMAC to improve robustness of the stego text.
\end{itemize}
\item Introduction
\begin{itemize}
\item There are three dimensions in a stego system:
\begin{enumerate}
    \item Payload Capacity: the ratio of hidden information to cover information.
    \item Robustness: the ability of the system to resist against changes in the cover object.
    \item Imperceptibility: the potential of the generated stego object to remain indistinguishable from other objects in the same category.
\end{enumerate}
\item The three dimensions often contradict each other.
\item Concealing any data in a text file is the most difficult kind of steganography due to the lack of redundant information in a text file.
\item Text steganography is broadly classified into two categories:
\begin{enumerate}
\item Linguistic Approach: the art of using written natural language to conceal secret messages.
\begin{itemize}
    \item Further divided into semantic and syntactic methods, with the syntactic method being divided into line-shift, word-shift, open-space and feature encoding.
\end{itemize}
\item Format-based Approach: uses physical formatting of text as a place in which to hide information.
\end{enumerate}
\end{itemize}
\item Linguistic Steganography
\begin{itemize}
\item The changes made to embed information in the cover text \textbf{do not result in an ungrammatical or unnatural text.}
\item Most methods of this type use either lexical (semantic) (ex. synonym substitution) or syntactic transformations (of the grammatical style of the original sentences) or a combination of the two.
\end{itemize}
\item Syntax of Language
\begin{itemize}
\item Set of rules that the language uses to combine words to create sentences.
\item Parts of speech combine into phrases.
\item Clauses can be broken down into phrases.
\item Sentences can have one or more independent or dependent clauses.
\end{itemize}
\item Proposed Approach
\begin{itemize}
\item The cover text (document in our case) is parsed by the parser to obtain a syntax tree, while the secret message (user identifer in our case) is compressed by the Shannon-Fano algorithm.
\item How do we get the syntax set of a sentence?
\end{itemize}
\end{enumerate}
\section{References}
\begin{enumerate}
    \item \href{https://bbengfort.github.io/2018/06/corenlp-nltk-parses/}{CoreNLP Setup}
    \item \href{https://stackoverflow.com/questions/42322902/how-to-get-parse-tree-using-python-nltk}{parse tree}
    \item \href{https://tex.stackexchange.com/questions/507288/minted-make-text-non-selectable-on-output-pdf}{Combat text selection}
    \item \href{https://stackoverflow.com/questions/28797418/replace-all-font-glyphs-in-a-pdf-by-converting-them-to-outline-shapes}{Combat text selection still}
    \item \href{https://lopezyse.medium.com/paraphrasing-in-natural-language-processing-nlp-857c28e68488}{read this on paraphrasing}
\end{enumerate}
\end{document}
